\documentclass[a4paper,12pt]{report}

\usepackage[italian]{babel}
\usepackage[utf8]{inputenc}
\usepackage{float}
\usepackage{hyperref}

\title{\textbf{Basi di dati}\\Nessuna responsabilità per quanto presente in questi appunti è attribuibile a terzi.}
\author{Luca Casadei}
\date{\today}

\begin{document}
	\maketitle
	\tableofcontents
	\chapter{Introduzione}
	\section{Definizioni principali}
	\textbf{Base di dati}: è una componente di un \textit{sistema informativo} atto a memorizzare in maniera strutturata dei \textit{dati}.\\
	\textbf{DBMS}: Un software capace di accedere in maniera efficace ed efficiente ai dati del database necessari al sistema informativo, segue un modello logico e garantisce \textit{persistenza}, \textit{condivisione}, \textit{affidabilità} e \textit{riservatezza}.
	\section{Applicazioni dei database}
	\begin{itemize}
		\item \textbf{OLTP}: Si usa generalmente un approccio transazionale in cui i dati sono recenti (non uno storico molto grande), con quantità di dati ridotta in cui le operazioni di scrittura sono frequenti quanto quelle di lettura.
		\item \textbf{OLAP}: Se si ha un interesse ad analizzare dati storici per soluzioni strategiche in futuro (es: per aggiungere una nuova categoria di prodotto una certa azienda potrebbe servire analizzare uno storico). Si può rappresentare questo tipo con dei Cubi, che facilitano alcune ricerche dei dati necessari. I contenuti di questi database sono di più alto livello, non scendono nei dettagli dell'elemento del database in se.
	\end{itemize}
A seconda delle operazioni da eseguire, possiamo scegliere quale tipo di modello DBMS usare, se di tipo \textit{relazionale} o \textit{non-relazionale}.
\subsection{Database \textit{non-relazionali} e loro applicazione}
Un'altra applicazione dei database \textit{non-relazionali} è nel contesto dell'intelligenza artificiale o della \textit{sentiment analisys}, in questo caso però i dati sono talmente numerosi che è conveniente utilizzare sistemi DBMS più flessibili che riescano a gestire anche elementi meno organizzati e strutturati che si auto-descrivono (come XML o JSON).\\
\subsubsection{Vantaggi dei database non-relazionali}
\begin{itemize}
	\item \textbf{Scalabilità}: Sono più scalabili e generalmente memorizzati su nodi in maniera distribuita, aggiungere nodi in questo caso non è un'operazione difficile.
	\item \textbf{Fault-tolerance}: Deve essere garantita la possibilità di accedere agli elementi anche se ci sono dei "guasti" in dei nodi del database (fault-tolerance maggiore).
	\item \textbf{Sharding (del dato) o parallelismo}: Si può lavorare in parallelo su più nodi per effettuare una ricerca più rapida distribuendo le elaborazioni. Per massimizzare le prestazioni si adotta un'organizzazione diversa, e non a tabelle come nei database \textit{relazionali}.
\end{itemize}
  
\subsection{Figure dei database \textit{relazionali}}
\begin{itemize}
	\item \textbf{DBA}: Database administrator, esso ha il compito di effettuare il monitoraggio, creare gli oggetti logici, analizzare le prestazioni ed ottimizzarle se necessario, gestire i permessi con i relativi utenti etc\dots
	\item \textbf{Database designer}: Si occupa di fornire il modello logico (progettazione) del database da implementare.
	\item \textbf{Software Engineer}: Si occupa di effettuare l'analisi che poi servirà al designer per progettare il database.
	\item \textbf{End User}: Si interfacciano indirettamente con il database.
\end{itemize}
\subsubsection{Classificazione degli end user}
\begin{itemize}
	\item \textbf{End User "naive"}: Sono utenti che non hanno alcun tipo della rappresentazione dei dati, ma vi accedono attraverso delle query preconfezionate e già programmate.
	\item \textbf{End User sofisticati}: Sono utenti che non partecipano attivamente alla progettazione del database ma vi hanno accesso e possono effettuarvi modifiche, devono quindi conoscere come sono organizzati i dati.
\end{itemize}
\section{Realizzare un database}
Nell'approccio transazionale in genere si utilizza il modello relazione, in questo contesto dobbiamo \textbf{progettare} uno schema \textit{Entità-Relazione}, questa rappresentazione ci consente di creare logicamente delle tabelle che contengono i nostri dati e le relazioni che vi sono tra le diverse tabelle. La progettazione è la parte più complessa da realizzare, le successive sono più meccaniche.\\
Come \textbf{accedere} ai contenuti del database che abbiamo creato? Si utilizza il linguaggio SQL, che si basa sul concetto dell'algebra relazionale.\\
Ci possono essere diversi approcci per l'accesso ai dati:
\begin{itemize}
	\item Approccio con API di accesso al database (es: JDBC per Java)
	\item Approccio ORM, in cui si associano le entità del database a degli oggetti di un linguaggio orientato agli oggetti, con un metodo che in genere è "code-first" e più vicino alla programmazione.
\end{itemize}
\section{Sistemi informativi}
\textbf{Dato}: Immediatamente presente alla conoscenza prima di essere elaborato.\\\\
\textbf{Informazione}: Quando si da un significato ad un dato, esso compone un'informazione, ogni dato deve quindi essere elaborato in genere.\\\\
Per elaborare un dato bisogna riconoscere delle classi partendo da un'istanza, che compone il dato, definendo quindi un dominio delle classi e delle istanze presenti nel dato.\\
Questa classificazione però non basta, bisogna anche capire che relazione vi è tra le classi che riusciamo a definire, cercando di risolvere le ambiguità che fanno inevitabilmente parte della lingua naturale. Questi concetti potrebbero essere quelli della residenza, impiego, utilizzo etc\dots\\
Le fonti non è detto che siano sufficienti per fare un certo tipo di assunzione.
Un sistema informativo deve provvedere alla raccolta e  classificazione dei dati, tutto va fatto attraverso determinate procedure ed è particolarmente complesso, esso non richiede esclusivamente conoscenze informatiche, ma anche aziendale, economico etc\dots
Un sistema informatico invece si occupa di trasmutare il dato in informazione.\\
I sistemi informativi possono essere strutturati, semi-strutturati o non-strutturati, quelli strutturati in genere sono quelli che si basano su semplici transazioni, le interrogazione sono preconfezionate. Qualsiasi gestionale fa parte di questa categoria.\\
A livello semi-strutturato invece troviamo qualcosa di meno preconfezionato, come può essere un \textit{ERP} aziendale per gestire le risorse, come i dipendenti, le materie prime, la contabilità etc\dots\\
A livello non strutturato invece troviamo i Decision Support System per effettuare delle decisioni strategiche per questioni di importanza rilevante, che non posso decidere effettuando una semplice query, ma devo fare delle valutazioni su una grande quantità di dati per fornire poche informazioni di sintesi per compiere queste decisioni.
Un sistema informativo è paragonabile ad un processo produttivo, in cui la materia grezza non ha valore, ma lo assume solo quando questa viene processata in informazione.
\subsection{Cosa rende un'informazione utile}
\begin{itemize}
	\item \textbf{Soggettività}: L'informazione è soggettiva, quello che interessa ad un soggetto non deve necessariamente interessare a tutta la collettività.
	\item \textbf{Rilevanza} relativa alla decisione da prendere.
	\item \textbf{Tempestività}: Non ci serve a nulla un'informazione se non è disponibile al momento della decisione
\end{itemize}
Tanti dati hanno un valore molto basso, questo dato acquisisce valore strategico quando è presente in una quantità molto piccola.
\subsection{Gestione dei dati con FileSystem}
Dall'applicazione devo implementare funzionalità di basso livello per leggere o scrivere i dati da un file, sfruttando le funzionalità di un sistema operativo, che non è impossibile a patto che si conosca la struttura adeguata per rappresentare entità e relazioni. Per certo tipo di applicazioni questo procedimento potrebbe essere più efficiente, ma richiede uno sforzo da parte del programmatore nettamente superiore rispetto alla gestione con DBMS, nelle applicazioni reali vi è la necessità però di gestire una quantità di nati molto grande, e la gestione va fatta in modo \textit{persistente} e \textit{condiviso} con anche accessi concorrenti, che venendo fatti su file non è molto semplice, stessa cosa per la gestione dei vincoli che vanno gestiti in maniera esplicita, una modifica ad un dato si potrebbero dover modificare una quantità eccessiva di file.\\
Sono esigenze che rendono complicata la gestione con i File.
Nelle aziende si avevano quindi sistemi specifici in ogni area, per lo scambio da un sistema ad un altro era complicato, perché ogni area rappresentava i dati in maniera diversa e spesso non compatibile o standardizzata tra le parti, vi era inoltre una ridondanza eccessiva dei dati, presenti in diverse aree senza avere una diretta interazione tra le parti.
\subsection{Vantaggi di un DBMS}
\begin{itemize}
	\item Dati persistenti e condivisi.
	\item Indipendenza tra programmi e dati, e tra programmi e operazioni.
	\item Persistenza e condivisione dei dati.
	contenuto...
\end{itemize}
\end{document}