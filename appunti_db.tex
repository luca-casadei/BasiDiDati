\documentclass[a4paper,12pt]{report}

\usepackage[italian]{babel}
\usepackage[utf8]{inputenc}
\usepackage{float}
\usepackage{hyperref}
\usepackage[italian]{cleveref}

\title{Basi di dati\\Appunti tratti dalle lezioni della prof.ssa. \textit{Annalisa Franco}}
\author{Luca Casadei}
\date{\today}

\begin{document}
	\maketitle
	\tableofcontents

	\chapter{Introduzione}
	\section{Definizioni principali}
	\textbf{Database:} Una collezione composta da dati di diverso genere.\\\\
	\textbf{DBMS}: Un software capace di accedere in maniera efficace ed efficiente ai dati del database.\\
	\section{Applicazioni dei database}
	\begin{itemize}
		\item OLTP: Si usa generalmente un approccio transazionale
		\item OLAP: Se si ha un interesse ad analizzare dati storici per soluzioni strategiche in futuro (es: per aggiungere una nuova categoria di prodotto una certa azienda potrebbe servire analizzare uno storico). Si può rappresentare questo tipo con dei Cubi, che facilitano alcune ricerche dei dati necessari. I contenuti di questi database sono di più alto livello, non scendono nei dettagli dell'elemento del database in se.
	\end{itemize}
A seconda delle operazioni da eseguire, possiamo scegliere quale tipo di modello DBMS usare, se di tipo \textit{relazionale} o \textit{non-relazionale}.
\subsection*{Database \textit{non-relazionali} e loro applicazione}
Un'altra applicazione dei database è nel contesto dell'intelligenza artificiale o della \textit{sentiment analisys}, in questo caso però i dati sono talmente numerosi che è conveniente utilizzare sistemi DBMS più flessibili che riescano a gestire anche elementi meno organizzati e strutturati.\\
Sono più scalabili e generalmente memorizzati su nodi in maniera distribuita, aggiungere nodi in questo caso non è un'operazione difficile.
Deve essere garantita la possibilità di accedere agli elementi anche se ci sono dei "guasti" in dei nodi del database (fault-tolerance maggiore). Si può lavorare in parallelo su più nodi per effettuare una ricerca più rapida. Per massimizzare le prestazioni si adotta un'organizzazione ad albero, e non a tabelle come nei database \textit{relazionali}.
\subsection{Figure dei database \textit{relazionali}}
\begin{itemize}
	\item \textbf{DBA}: Database administrator, esso ha il compito di effettuare il monitoraggio, creare gli oggetti logici, analizzare le prestazioni ed ottimizzarle se necessario, gestire i permessi con i relativi utenti etc\dots
	\item \textbf{Database designer}: Si occupa di fornire il modello logico (progettazione) del database da implementare.
	\item \textbf{Software Engineer}: Si occupa di effettuare l'analisi che poi servirà al designer per progettare il database.
	\item \textbf{End User}: Si interfacciano indirettamente con il database.
\end{itemize}
\subsubsection{Classificazione degli end user}
\begin{itemize}
	\item \textbf{End User "naive"}: Sono utenti che non hanno alcun tipo della rappresentazione dei dati, ma vi accedono attraverso delle query preconfezionate e già programmate.
	\item \textbf{End User sofisticati}: Sono utenti che non partecipano attivamente alla progettazione del database ma vi hanno accesso e possono effettuarvi modifiche, devono quindi conoscere come sono organizzati i dati.
\end{itemize}
\section{Realizzare un database}
Nell'approccio transazionale in genere si utilizza il modello relazione, in questo contesto dobbiamo \textbf{progettare} uno schema \textit{Entità-Relazione}, questa rappresentazione ci consente di creare logicamente delle tabelle che contengono i nostri dati e le relazioni che vi sono tra le diverse tabelle. La progettazione è la parte più complessa da realizzare, le successive sono più meccaniche.\\
Come \textbf{accedere} ai contenuti del database che abbiamo creato? Si utilizza il linguaggio SQL, che si basa sul concetto dell'algebra relazionale.\\
Ci possono essere diversi approcci per l'accesso ai dati:
\begin{itemize}
	\item Approccio con API di accesso al database (es: JDBC per Java)
	\item Approccio ORM, in cui si associano le entità del database a degli oggetti di un linguaggio orientato agli oggetti, con un metodo che in genere è "code-first" e più vicino alla programmazione.
\end{itemize}
\end{document}